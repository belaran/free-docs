%%%%%%%%%%%%%%%%%%%%%%%%%%%%%%%%%%%%%%%%%%%%%%%%%%%%%%%%%%%%%%%%%%%%%%%%
%                                                                      %
% This program is free software; you can redistribute it and/or modify %
% it under the terms of the GNU General Public License as published by %
% the Free Software Foundation; either version 2 of the License, or    %
% (at your option) any later version.                                  %
%                                                                      %
% This program is distributed in the hope that it will be useful,      %
% but WITHOUT ANY WARRANTY; without even the implied warranty of       %
% MERCHANTABILITY or FITNESS FOR A PARTICULAR PURPOSE.  See the        %
% GNU General Public License for more details.                         %
%                                                                      %
% You should have received a copy of the GNU General Public License    %
% along with this program; if not, write to the Free Software          %
% Foundation, Inc., 51 Franklin St, Fifth Floor, Boston,               %
% MA  02110-1301  USA                                                  %
%                                                                      %
%%%%%%%%%%%%%%%%%%%%%%%%%%%%%%%%%%%%%%%%%%%%%%%%%%%%%%%%%%%%%%%%%%%%%%%%
%
%	$Id$
%

\setcounter{remarque-cnt}{1}
\setcounter{example-cnt}{1}
\chapter{\label{adv-fltrs-sed}Utilisation de <<~{\tt sed}~>>}
\thispagestyle{fancy}

%%%%%%%%%%%%%%%%%%%%%%%%%%%%%%%%%%%%%%%%%%%%%%%%%%%%%%%%%%%%
\section{Introduction}

La commande \index{sed@\texttt{sed}}<<~{\tt sed}~>> est un {\'e}diteur non interactif (comme peuvent
l'{\^e}tre <<~\index{vi@\texttt{vi}}{\tt vi(1)}~>> et <<~{\tt ed(1)}~>>). Elle est utilis{\'e}e
g{\'e}n{\'e}ralement comme filtre.

\begin{definition}{Syntaxe}
\verb=sed [-n requ{\^e}te] [-e requ{\^e}te]=~$\cdots$~\verb=[-f fichier.requ{\^e}tes] [fichiers]=~$\cdots$
\end{definition}

\begin{definition}{Description des options}
\begin{tabular}{l@{\hspace{2ex}}p{8cm}}
	{\tt -e} {\it requ{\^e}te}	&
		Cette option indique que la requ{\^e}te fait partie de la ligne de
		commandes. Il est possible de faire appel plusieurs fois {\`a} cette
		option pour un m{\^e}me appel {\`a} la commande <<~{\tt sed}~>>.
		Les requ{\^e}tes seront trait{\'e}es s{\'e}quentiellement pour chaque ligne
		lues sur l'entr{\'e}e standard ou sur les fichiers sp{\'e}cifi{\'e}s en
		argument.	\\[1ex]
	{\tt -f} {\it fichier.requ{\^e}tes}	&
		Cette option permet d'utiliser un fichier comme texte de
		requ{\^e}tes.	\\[1ex]
	{\tt -n} {\it requ{\^e}te}	&
		Cette option permet d'emp{\^e}cher l'impression des lignes trait{\'e}es.
					\\[1ex]
\end{tabular}
\end{definition}

\begin{remarque}
Les requ{\^e}tes sp{\'e}cifi{\'e}es avec les options <<~{\tt -n}~>> ou <<~{\tt -e}~>>
peuvent {\^e}tre prot{\'e}g{\'e}es gr{\^a}ce aux caract{\`e}res <<~{\tt "}~>> et <<~{\tt '}~>>
pour {\'e}viter l'interpr{\'e}tation par le Shell.
\end{remarque}

\begin{example}
\begin{verbatim}
sed -e 'requ1' -e 'req2' fichier.in > fichier.out
\end{verbatim}
\end{example}

%%%%%%%%%%%%%%%%%%%%%%%%%%%%%%%%%%%%%%%%%%%%%%%%%%%%%%%%%%%%
\section{\label{sed-fonct}Mode de fonctionnement}

La commande \index{sed@\texttt{sed}!fonctionnement}<<~{\tt sed}~>> permet de copier les fichiers indiqu{\'e}s sur la
sortie standard en l'{\'e}ditant en fonction du texte de requ{\^e}tes.
Si aucun fichier n'est pr{\'e}cis{\'e} en entr{\'e}e, <<~{\tt sed}~>> lit sur son entr{\'e}e
standard.

La sortie standard peut {\^e}tre redirig{\'e}e sur un fichier avec les
m{\'e}canismes classiques du Shell (cf. section \ref{redirect-io}). Les
requ{\^e}tes de la commande <<~{\tt sed}~>> dont identiques {\`a} celles de la
commande <<~{\tt ed(1)}~>>. La r{\'e}p{\'e}tition des requ{\^e}tes globales de
substitution est plus efficace avec <<~{\tt sed}~>> qu'avec <<~{\tt
ed}~>>.

Pendant son ex{\'e}cution, la commande <<~{\tt sed}~>> proc{\`e}de de la fa\c{c}on
suivante~:
\begin{enumerate}
	\item	Chaque ligne s{\'e}lectionn{\'e}e par les adresses est copi{\'e}e dans un
			espace de travail (<<~{\sl pattern space}~>>) et toutes les
			commandes lui sont appliqu{\'e}es.
	\item	Lorsque toutes les commandes ont {\'e}t{\'e} appliqu{\'e}es, le contenu
			de l'espace de travail modifi{\'e} est envoy{\'e} sur la sortie
			standard. Le contenu de l'espace de travail est supprim{\'e} afin
			de recevoir un nouveau contenu.
	\item	En plus de l'espace de travail, <<~{\tt sed}~>> dispose d'un espace m{\'e}moire
			(<<~{\sl hold space}~>>) o{\`u} peut {\^e}tre stock{\'e} temporairement
			le contenu de l'espace de travail. L'utilisation de cet espace
			est utilis{\'e} par les commandes <<~{\tt g}~>>, <<~{\tt G}~>>,
			<<~{\tt h}~>> et <<~{\tt H}~>> d{\'e}crites {\`a} la section
			\ref{sed-cmds}.
\end{enumerate}

%%%%%%%%%%%%%%%%%%%%%%%%%%%%%%%%%%%%%%%%%%%%%%%%%%%%%%%%%%%%
\section{Formulation des requ{\^e}tes}

%%%%%%%%%%%%
\subsection{Introduction}

La forme g{\'e}n{\'e}rale d'une \index{sed@\texttt{sed}!requ{\^e}te}requ{\^e}te est~:
\begin{center}
\begin{verbatim}
[adresse1],[adresse2]commande[argument]
\end{verbatim}
\end{center}

Les adresses sp{\'e}cifi{\'e}es dans une requ{\^e}tes d{\'e}signent les limites
(d{\'e}but et fin) o{\`u} doit s'appliquer la  commande. L'argument repr{\'e}sente
une cha{\^i}ne de caract{\`e}res, ou du texte ou bien encore un fichier suivant les
commandes utilis{\'e}es.

%%%%%%%%%%%%
\subsection{\label{sed-def-addr}D{\'e}finition des adresses}

Les adresses peuvent {\^e}tre d{\'e}crites de deux fa\c{c}ons~:
\begin{itemize}
	\item	soit par des valeurs num{\'e}riques donnant les num{\'e}ros de lignes
			de l'espace de travail ou le caract{\`e}re <<~\verb=$=~>>
			indiquant sa derni{\`e}re ligne,
	\item	soit par des expressions r{\'e}guli{\`e}res plac{\'e}es entre deux
			{\sl slash} <<~{\tt /}~>>.
\end{itemize}
Une requ{\^e}te pr{\'e}c{\'e}d{\'e}e par deux adresses  s'applique {\`a} toutes les lignes
comprises entre ces deux adresses. Dans le cas o{\`u} deux adresses seraient
identiques, il suffit de la sp{\'e}cifier seulement une fois. Si aucune
adresse n'est sp{\'e}cifi{\'e}e, l'espace de travail correspond {\`a} la totalit{\'e}
des lignes en entr{\'e}e. La commande <<~{\tt sed}~>> est donc appliqu{\'e}e {\`a}
toutes les lignes.

\begin{definition}{Syntaxes}
{\tt {\sl Adresse1},{\sl Adresse2Commande}\\
{\sl Adresse1},/{\sl Expression$_{r\acute{e}guli\grave{e}re}$}/{\sl Commande}\\
/{\sl Expression$_{r\acute{e}guli\grave{e}re}$}/,{\sl Adresse2Commande}\\
/{\sl Expression$^1_{r\acute{e}guli\grave{e}re}$}/,/{\sl Expression$^2_{r\acute{e}guli\grave{e}re}$}/{\sl Commande}\\
{\sl AdresseCommande}\\
/{\sl Expression$_{r\acute{e}guli\grave{e}re}$}/{\sl Commande}\\
{\sl Commande}
}
\end{definition}

\begin{example}
On utilise ici, {\`a} titre d'exemple pour les adresses, la commande
<<~{\tt d}~>>, permettant de d{\'e}truire les lignes correspondantes
de l'espace de travail.

\begin{tabular}{l@{\hspace{2ex}}p{8cm}}
	\verb=10,20d=		&
		D{\'e}truit les lignes comprises entre la dixi{\`e}me et la vingti{\`e}me
		ligne .	\\[1ex]
	\verb=/^#/,$d=		&
		D{\'e}truit les lignes comprises entre la premi{\`e}re ligne commen\c{c}ant
		par <<~{\tt \#}~>> jusqu'{\`a} la derni{\`e}re ligne du fichier
		(num{\'e}ro de ligne <<~\verb=$=~>>).	\\[1ex]
	\verb=/^#/,/end$/d=	&
		D{\'e}truit les lignes comprises entre la premi{\`e}re ligne commen\c{c}ant
		par <<~{\tt \#}~>> et la premi{\`e}re ligne se terminant par
		<<~{\tt end}~>>.	\\[1ex]
	\verb=10d=			&
		D{\'e}truit la dixi{\`e}me ligne.	\\[1ex]
	\verb=/10/d=		&
		D{\'e}truit toute ligne contenant la cha{\^\i}ne <<~{\tt 10}~>>.	\\[1ex]
	\verb=d=			&
		D{\'e}truit l'ensemble de l'espace de travail.\\
\end{tabular}
\end{example}

%%%%%%%%%%%%
\subsection{\label{sed-cmds}Les Commandes}

La liste des \index{sed@\texttt{sed}!commandes}commandes cit{\'e}es dans ce paragraphe font parties des plus
utilis{\'e}es. Cette liste n'est pas exhaustive. Pour plus de renseignements,
reportez vous au manuel des commandes <<~{\tt sed(1)}~>> et
<<~{\tt ed(1)}~>>.

\begin{tabular}
	{|@{\hspace{1ex}}l@{\hspace{1ex}}|@{\hspace{1ex}}p{8cm}@{\hspace{1ex}}|}
	\hline
	Commande	&	Description	\\
	\hline
	\hline
		{\tt p}				&
			Copie le contenu de l'espace de travail sur la sortie
			standard.\\[1ex]
		{\tt a}{\it texte}	&
			Ajoute du texte apr{\`e}s la position courante.	\\[1ex]
		{\tt i}{\it texte}	&
			Ajoute du texte avant la position courante.	\\[1ex]
		{\tt c}{\it texte}	&
			Change le texte de la ligne courant par <<~{\it texte}~>>.	\\[1ex]
		{\tt s/}{\it expr$_1$}{\tt /}{\it expr$_2$}{\tt /}$[${\tt g}$]$	&
			Substitue la premi{\`e}re chaine satisfaisant l'expression
			r{\'e}guli{\`e}re 1 ({\it expr$_1$}) par le texte correspondant {\`a}
			l'expression r{\'e}guli{\`e}re 2 ({\it expr$_2$}). L'option
			<<~{\tt g}~>> signifie que la substitution doit se
			faire autant de fois que n{\'e}cessaire sur la ligne trait{\'e}e.
			En effet, par d{\'e}faut, la commande <<~{\tt s}~>> ne s'applique
			qu'{\`a} la premi{\`e}re chaine satisfaisant l'expression
			r{\'e}guli{\`e}re dans la ligne courante de l'espace de travail.
			Avec l'option <<~{\tt g}~>>, toute chaine satisfaisant
			l'expression r{\'e}guli{\`e}re <<~{\it expr$_1$}~>> dans la
			ligne courante de l'espace de travail sera remplac{\'e}e.	\\[1ex]
		{\tt d}			&
			D{\'e}truit la ligne courante.	\\[1ex]
		{\tt g}			&
			Remplace le contenu de l'espace de travail ({\sl pattern space})
			par le contenu de l'espace m{\'e}moire ({\sl holder space}).\\[1ex]
		{\tt G}			&
			Ajoute le contenu de l'espace m{\'e}moire ({\sl holder space})
			au contenu de l'espace de travail ({\sl pattern space}).\\[1ex]
		{\tt h}			&
			Remplace le contenu de l'espace m{\'e}moire ({\sl holder space})
			par le contenu de l'espace de travail ({\sl pattern space}).\\[1ex]
		{\tt H}			&
			Ajoute le contenu de l'espace de travail ({\sl pattern space})
			au contenu de l'espace m{\'e}moire ({\sl holder space}).	\\[1ex]
		\verb*=r ={\it fichier}	&
			Ins{\`e}re le contenu d'un fichier apr{\`e}s la ligne courante.	\\[1ex]
		\verb*=w ={\it fichier}	&
			Met le contenu de l'espace de travail dans un fichier.	\\
	\hline
\end{tabular}

\begin{example}
\verb*=sed -n "/lancelot/p" donjons.dragons=\\[0.5cm]
Copie toutes les lignes de l'espace de travail contenant la chaine
<<~{\tt lancelot}~>>.\\[0.5cm]

\verb*=sed -e '/lancelot/a et arthur' donjons.dragons=\\[0.5cm]
Ins{\`e}re la chaine <<~\verb*= et arthur=~>>\footnote{Notez les espaces
dans la chaine <<~{\tt $_\sqcup$et$_\sqcup$arthur}~>>.} apr{\`e}s la chaine
<<~{\tt lancelot}~>> sur la totalit{\'e} de l'espace de travail.\\[0.5cm]

\verb*=sed -e '/dulac/ilancelot ' donjons.dragons=\\[0.5cm]
Ins{\`e}re la chaine <<~\verb*=lancelot =~>>\footnote{Notez l'espace
suivant la chaine <<~{\tt lancelot}~>>.} avant la chaine <<~{\tt dulac}~>>
sur la totalit{\'e} de l'espace de travail.\\[0.5cm]

\verb*=sed -e 's/Salut/Bonjour/g' welcome.txt=\\[0.5cm]
Substitue <<~{\tt Salut}~>> par <<~{\tt Bonjour}~>> dans la totalit{\'e}
du fichier <<~{\tt welcome.txt}~>>.
\end{example}


%%%%%%%%%%%%
\subsection{Les symboles particuliers}

Les \index{sed@\texttt{sed}!symboles}symboles particuliers sont tous ceux des expressions r{\'e}guli{\`e}res {\`a}
l'exception des symboles <<~{\tt +}~>> et <<~{\tt |}~>>. A cette liste,
on rajoute les symboles de la commande de substitution <<~{\tt s}~>>.
Ces symboles sont~:

\begin{center}
\begin{tabular}{|@{\hspace{2ex}}l@{\hspace{2ex}}|@{\hspace{2ex}}p{7cm}@{\hspace{2ex}}|}
	\hline
	Symbole	&	Description	\\
	\hline \hline
	<<~\verb=\(=~>> et <<~\verb=\)=~>>	&
		D{\'e}finit une portion d'expression r{\'e}guli{\`e}re.	\\[0.5ex]
	\hline
	<<~\verb=\=$n$~>>					&
		Ne fait pas partie des expressions r{\'e}guli{\`e}res mais repr{\'e}sente
			la $n^{i\grave{e}me}$ portion d'une expression r{\'e}guli{\`e}re.
			\\[0.5ex]
	\hline
	<<~\verb=&=~>>						&
		Ne fait pas partie des expressions r{\'e}guli{\`e}res mais repr{\'e}sente
		la portion de ligne d'entr{\'e}e qui correspond {\`a} l'expression
		r{\'e}guli{\`e}re.
		\\
	\hline
\end{tabular}
\end{center}

\begin{example}
\begin{verbatim*}
s/\([ \t]*function[ \t]*\)\([0-z]*\)\([ \t]*(.*)\)/#FONCTION \2/
\end{verbatim*}

La commande <<~{\tt sed}~>> utilis{\'e}e ici est la commande <<~{\tt s}~>>. Sa
syntaxe est~:
\begin{center}
$[${\sl adresse1}$[$,{\sl adresse2}$]${\tt s/}{\it expr$_1$}{\tt /}{\it expr$_2$}{\tt /}$[${\tt g}$]$
\end{center}

On constate donc que les adresses de d{\'e}but et de fin sont absentes. Par
cons{\'e}quent, cette requ{\^e}te s'applique {\`a} toutes les lignes en entr{\'e}e
(cf. \ref{sed-def-addr}). Apr{\`e}s analyse, il appara{\^\i}t que
cette requ{\^e}te <<~{\tt sed}~>> se d{\'e}compose de la fa\c{c}on suivante~:\\[1ex]
{\tt s/}\\
\hspace{0.5cm}\verb*=\([ \t]*function[ \t]*\)=\\
\hspace{0.5cm}\verb*=\([0-z]*\)=\\
\hspace{0.5cm}\verb*=\([ \t]*(.*)\)=\\
{\tt /}\\
\hspace{0.5cm}\verb*=#FONCTION \2=\\
{\tt /}\\[1ex]
Nous distinguons donc trois regroupements d'expression r{\'e}guli{\`e}re~:
\begin{enumerate}
	\item	<<~\verb*=[ \t]*funtion[ \t]*=~>>
	\item	<<~\verb=[0-z]*=~>>
	\item	<<~\verb*=[ \t]*(.*)=~>>
\end{enumerate}
Dans le second membre de la commande <<~{\tt s}~>> (<<~\verb=#FONCTION \2=~>>)
l'expression <<~\verb=\2=~>> r{\'e}f{\'e}rence donc la chaine correspondant {\`a}
<<~\verb=[0-z]*=~>>.

Analysons chaque regroupement du premier membre de cette requ{\^e}te~:
<<~\verb*=[ \t]*funtion[ \t]*=~>>.

Tout d'abord <<~\verb*=[ \t]*=~>> d{\'e}signe le caract{\`e}re
\spacekey ou bien \tabkey un nombre quelconque de fois. <<~{\tt
function}~>> repr{\'e}sente tout simplement cette chaine de caract{\`e}res. Par
cons{\'e}quent, <<~\verb*=[ \t]*funtion[ \t]*=~>> d{\'e}signe la chaine <<~{\tt
function}~>> suivie et pr{\'e}c{\'e}d{\'e}e d'espaces et de tabulations.

La seconde partie, <<~\verb=[0-z]*=~>>, d{\'e}signe un des caract{\`e}res
dont le code {\ASCII} est compris entre celui de <<~{\tt 0}~>> et celui
de <<~{\tt z}~>>\footnote{Cette plage inclue l'ensemble des chiffres,
toutes les lettres minuscules et majuscules, plus un certain nombre de
caract{\`e}res de ponctuation. Pour plus de pr{\'e}cisions, reportez-vous
{\`a} la table du jeu de caract{\`e}res {\ASCII}.}, un nombre quelconque de fois.
Nous avons donc, approximativement, une chaine de caract{\`e}re quelconque
{\bf sans espaces ni tabulations}.

La troisi{\`e}me partie, <<~\verb*=[ \t]*(.*)=~>> reprend la s{\'e}quence
<<~\verb*=[ \t]*=~>> repr{\'e}sentant une s{\'e}quence d'espaces et de tabulations
un nombre quelconque de fois. <<~\verb*=(.*)=~>> d{\'e}signe le caract{\`e}re
<<~{\tt (}~>>, suivi de caract{\`e}res quelconques un nombre ind{\'e}termin{\'e} de fois
(<<~\verb=.*=~>>), suivi par le caract{\`e}re <<~{\tt )}~>>. En conclusion,
cette derni{\`e}re partie repr{\'e}sente une chaine de caract{\`e}res quelconque
({\'e}vuentuellement vide) encadr{\'e}e par des parenth{\`e}ses et pr{\'e}c{\'e}d{\'e}e par un
nombre quelconque d'espaces et de tabulations.

Le premier membre de cette requ{\^e}te <<~{\tt sed}~>>,
\begin{center}
\begin{verbatim*}
\([ \t]*function[ \t]*\)\([0-z]*\)\([ \t]*(.*)\)
\end{verbatim*}
\end{center}
repr{\'e}sente donc~:
\begin{enumerate}
	\item	le mot <<~{\tt function}~>> pr{\'e}c{\'e}d{\'e} et suivi d'un nombre
			quelconque d'espaces et de tabulations,
	\item	une chaine de caract{\`e}res quelconques {\bf sans espaces ni
			tabulations},
	\item	une chaine de caract{\`e}res entre parenth{\`e}ses pr{\'e}c{\'e}d{\'e}e te suivie
			d'un nombre quelconque d'espaces et de tabulations.
\end{enumerate}

La seconde partie de la requ{\^e}te <<~{\tt sed}~>>,
\begin{center}
\begin{verbatim*}
#FONCTION \2
\end{verbatim*}
\end{center}
substitue l'ensemble de la ligne par~:
\begin{itemize}
	\item	la chaine <<~\verb*=#FONCTION =~>>,
	\item	la partie correspondant au second membre de l'expression
			r{\'e}guli{\`e}re pr{\'e}c{\'e}dente, c'est-{\`a}-dire une chaine de caract{\`e}res
			quelconques {\bf sans espaces ni tabulations}.
\end{itemize}
\end{example}

\begin{example}
\begin{verbatim}
s/abcdef1234567890/&ghijkl/
\end{verbatim}

Cette requ{\^e}te <<~{\tt sed}~>> {\'e}quivaut {\`a}~:
\begin{verbatim}
s/abcdef1234567890/abcdef1234567890ghijkl/
\end{verbatim}
\end{example}

%%%%%%%%%%%%%%%%%%%%%%%%%%%%%%%%%%%%%%%%%%%%%%%%%%%%%%%%%%%%
\section{Exemples avanc{\'e}s}

Pour plus d'exemples sur la commande <<~{\tt sed}~>>, reportez vous au
chapitre \ref{adv-programming}.


%%%%%%%%%%%%%%
\subsection{\label{adv-fltrs-sed-ex1}Exemple 1}

Le but de cet exemple est de~:
\begin{description}
	\item[But~:]\mbox{}\\
		L'utilisateur d{\'e}sire conna{\^\i}tre la localisation dans l'arborescence du
		syst{\`e}me de diverses commandes.\\[2ex]
	\item[R{\'e}alisation~:]\mbox{}\\
		On se propose, alors, de d{\'e}velopper un script en Bourne Shell, dont le nom est
		<<~{\tt locate}~>>, acceptant un nombre quelconque d'arguments
		repr{\'e}sentant les commandes {\`a} localiser et d'afficher le chemin absolu de
		chaque ex{\'e}cutable correspondant. Si, par hasard, aucun fichier n'a pu {\^e}tre
		trouv{\'e}, on affichera un message d'erreur.\\[2ex]
	\item[Syntaxe~:]\mbox{}\\
		La syntaxe propos{\'e}e, pour appeler ce script est~:
		\begin{quote}
		{\tt locate}~{\sl file}$\cdots$
		\end{quote}
\end{description}

Le script obtenu est alors~:
\begin{verbatim}
#!/bin/sh
if [ $# -eq 0 ]; then
    echo "`basename $0`: missing arguments." >&2
    echo "usage: `basename $0` file ..." >&2
    exit -1
fi
while
    [ $# -ne 0 ]
do
    find_it=0
    for dir in `echo $PATH | sed -e 's/:/ /g'`
    do
        if [ -f $dir/$1 ]; then
            echo $dir/$1
            find_it=1
            break
        fi
    done
    [ $find_it -eq 0 ] && echo "$1 not found in \"PATH\" variable." >&2
    shift
done
\end{verbatim}

Nous allons maintenant d{\'e}tailler le fonctionnement.

\begin{verbatim}
#!/bin/sh
if [ $# -eq 0 ]; then
    echo "`basename $0`: missing arguments." >&2
    echo "usage: `basename $0` file ..." >&2
    exit -1
fi
\end{verbatim}
\begin{quote}

Tout d'abord, le script v{\'e}rifie que le nombre d'arguments est bien non nul gr{\^a}ce
{\`a} la variable <<~\verb,$#,~>>. Si ce n'est pas le cas, on extrait le nom du
script, contenu dans la variable <<~\verb,$0,~>> pour afficher le message d'erreur
correspondant, ainsi que la fa\c{c}on de l'utiliser.
\end{quote}

\begin{verbatim}
while
    [ $# -ne 0 ]
do
\end{verbatim}
\vspace{2ex}
$\cdots$
\\[2ex]
\begin{verbatim}
    shift
done
\end{verbatim}
\begin{quote}
Afin de pouvoir analyser tous les arguments, le programme va effectuer une boucle
tant que le nombre d'arguments est non nul. L'utilisation de la commande <<~{\tt shift}~>>
permettra de mettre {\`a} jour la valeur qui y est contenue et l'argument qui sera trait{\'e} sera
toujours contenu dans la variable positionnelle <<~{\tt 1}~>> (son contenu sera appel{\'e} gr{\^a}ce
{\`a} <<~\verb=$1=~>>).
\end{quote}

\begin{verbatim}
find_it=0
for dir in `echo $PATH | sed -e 's/:/ /g'`
do
\end{verbatim}
\vspace{2ex}
$\cdots$
\\[2ex]
\begin{verbatim}
done
\end{verbatim}
\begin{quote}
La variable <<~{\tt PATH}~>> contient la liste des r{\'e}pertoires {\`a} examiner, sachant que
chaque nom est s{\'e}par{\'e} par le caract{\`e}re <<~{\tt :}~>>. La boucle <<~{\tt for}~>>
admet une liste de valeurs s{\'e}par{\'e}es par des espaces. Par cons{\'e}quent, il faut fournir
{\`a} <<~{\tt for}~>>, cette liste {\`a} partir du contenu de <<~{\tt PATH}~>> dont il faudra
remplacer <<~{\tt :}~>> par \spacekey.

Pour cela, la commande <<~\verb,echo $PATH,~>> renvoie sur l'entr{\'e}e standard de la
commande <<~{\tt sed}~>>. La requ{\^e}te <<~\verb*,s/:/ /g,~>> substitue chaque
occurence (gr{\^a}ce {\`a} l'option <<~{\tt g}~>>) du caract{\`e}re <<~{\tt :}~>> par
\spacekey.

Le r{\'e}sultat de <<~\verb*,`echo $PATH | sed -e 's/:/ /g'`,~>> est {\'e}valu{\'e} par le
shell lors de l'{\'e}tape des <<~{\sl substitutions de commandes}~>>. Le r{\'e}sultat obtenu
est donc~:
\begin{center}
\verb,for dir in, {\sl rep1} {\sl rep2} {\sl rep3} $\cdots$
\end{center}
Par cons{\'e}quent, la variable <<~{\tt dir}~>> va contenir, {\`a} chaque it{\'e}ration, l'un
des r{\'e}pertoires contenus dans la variable <<~{\tt PATH}~>>. Il ne restera plus qu'{\`a}
v{\'e}rifier l'existance du fichier concern{\'e}, dont le nom, lui, est contenu dans la variable
<<~{\tt 1}~>>. Cette variable prendra, {\`a} tour de r{\^o}le, la valeur des diff{\'e}rents
arguments pass{\'e} au script, gestion assur{\'e} par la boucle <<~{\tt while}~>> associ{\'e}
{\`a} la commande <<~{\tt shift}~>>.

La variable <<~{\tt find\_it}~>> est un {\it flag} indiquant si le fichier a {\'e}t{\'e} trouv{\'e}
lors de l'ex{\'e}cution de la boucle <<~{\tt for}~>>.
\end{quote}

\begin{verbatim}
if [ -f $dir/$1 ]; then
    echo $dir/$1
    find_it=1
    break
fi
\end{verbatim}
\begin{quote}
Cette op{\'e}ration est simple. Un test v{\'e}rifie l'existance du fichier. S'il existe, le nom complet
est affich{\'e} sur la sortie standard et le {\it flag} <<~{\tt find\_it}~>>
est positionn{\'e} {\`a} la valeur $1$ et on force la sortie de la boucle <<~{\tt for}~>>. Dans le
cas contraire, l'ex{\'e}cution de la boucle se poursuit.
\end{quote}

\begin{verbatim}
[ $find_it -eq 0 ] && echo "$1 not found in \"PATH\" variable." >&2
shift
\end{verbatim}
\begin{quote}
{\`A} ce stade de l'ex{\'e}cution, chaque r{\'e}pertoire contenu dans la variable <<~{\tt PATH}~>>
a {\'e}t{\'e} examin{\'e}, {\`a} moins qu'une sortie n'aie {\'e}t{\'e} provoqu{\'e} lorsque le fichier a {\'e}t{\'e} trouv{\'e}.
Dans ce cas, le script affichera le message d'information sur la sortie d'erreur
standard (redirection gr{\^a}ce {\`a} <<~\verb=>&2=~>>) {\`a} condition que la variable
<<~\verb=find_it=~>> soit non nulle.

Il ne restera plus qu'{\`a} passer {\`a} l'argument suivant gr{\^a}ce {\`a} la commande <<~\verb=shift=~>>.
\end{quote}


%%%%%%%%%%%%%%
\subsection{\label{adv-fltrs-sed-ex2}Exemple 2}

Le but de cet exemple est d'afficher toutes les lignes d'un fichier se terminant par
<<~\verb=*/=~>>.

Pour cela, nous allons utiliser la notion des espaces de travail de <<~{\tt sed}~>>.
Nous avons vu, {\`a} la section \ref{sed-fonct}, que <<~{\tt sed}~>> copiait dans son
espace de travail, toutes les lignes (ou {\sl enregistrements}), arrivant sur son
entr{\'e}e standard, correspondant aux crit{\`e}res de s{\'e}lection sp{\'e}cifi{\'e}s dans les adresses
de d{\'e}but et de fin de la requ{\^e}te.

Il ne nous reste donc qu'{\`a} formuler le crit{\`e}re de s{\'e}lection ad{\'e}quat et d'afficher sur
la sortie standard, le contenu de l'espace de travail.

La cha{\^\i}ne {\`a} consid{\'e}rer est donc <<~\verb=*/=~>>. Or le caract{\`e}re <<~\verb=*=~>> a une
signification particuli{\`e}re dans une expression r{\'e}guli{\`e}re. Par cons{\'e}quent, le caract{\`e}re
d'{\'e}chappement <<~\verb=\=~>> doit {\^e}tre utilis{\'e}. L'expression r{\'e}guli{\`e}re
correspondant {\`a} la cha{\^\i}ne <<~\verb=*/=~>> est donc <<~\verb=\*/=~>>. Il ne reste
plus qu'{\`a} y ajouter le caract{\`e}re ad{\'e}quat pour les expressions r{\'e}guli{\`e}res caract{\'e}risant
la fin d'une ligne, c'est-{\`a}-dire <<~\verb=$=~>>. Nous obtenons donc l'expression
suivante~:
\begin{center}
\verb=\*\/$=
\end{center}

Le crit{\`e}re de s{\'e}lection {\`a} appliquer dans la sp{\'e}cification des adresses de d{\'e}but et
de fin pour la requ{\^e}te <<~{\tt sed}~>> correspond {\`a} toutes les lignes satisfaisant
l'expression r{\'e}guli{\`e}re pr{\'e}c{\'e}dente. Ces deux adresses seront donc identiques et
correspondront {\`a} cette expression. Il suffira donc de ne la sp{\'e}cifier qu'une seule
fois en la d{\'e}limitant par le caract{\`e}re <<~\verb=/=~>>.

La commande permettant d'afficher l'espace de travail sur la sortie standard, est
<<~{\tt p}~>>.

Nous obtenons donc~:
\begin{quote}
\begin{verbatim}
sed -n '/\*\/$/p' fichier
\end{verbatim}
\end{quote}

%%%%%%%%%%%%%%
\subsection{\label{adv-fltrs-sed-ex3}Exemple 3}

Le but de cet exemple est de~:
\begin{itemize}
	\item	supprimer les lignes commen\c{c}ant par <<~\verb=#=~>>,
	\item	sur une ligne, tout ce qui suit le caract{\`e}re <<~\verb=#=~>>.
\end{itemize}

Nous devons donc proc{\'e}der en deux {\'e}tapes~:
\begin{description}
	\item[{\sl Premi{\`e}re {\'e}tape~:}]\mbox{}\\
		Le premier traitement {\`a} effectuer est celui de supprimer toute ligne
		comman\c{c}ant par <<~\verb=#=~>>. En effet, si notre commande supprime
		tout d'abord tout ce qui suit le caract{\`e}re <<~\verb=#=~>>, y compris
		celui-ci, nous obtiendrons une ligne blanche {\`a} supprimer, qu'il ne sera pas
		possible de distinguer d'une ligne d{\'e}j{\`a} blanche qui, elle, ne devra pas {\^e}tre
		supprim{\'e}e.
	\item[{\sl Seconde {\'e}tape~:}]\mbox{}\\
		Le traitement doit s'effectuer sur le r{\'e}sultat de la premi{\`e}re {\'e}tape. Nous obtenons
		une s{\'e}rie d'enregistrements ({\sl lignes de fichiers}) dont, aucune ne commence
		par <<~\verb=#=~>>. Il ne restera donc plus qu'{\`a} substituer tout ce qui suit le
		caract{\`e}re <<~\verb=#=~>> par rien, caract{\`e}re <<~\verb=#=~>> compris.
		Ce traitement s'effectuera sur toutes les lignes. L'op{\'e}ration de substitution ne
		s'ex{\'e}cutera donc seulement si il existe une occurence du caract{\`e}re <<~\verb=#=~>>
		sur la ligne.
\end{description}

Deux m{\'e}thodes sont possibles~:
\begin{itemize}
	\item	soit nous utilisons deux commandes <<~{\tt sed}~>> disposant chacune
			de la requ{\^e}te associ{\'e}e {\`a} l'{\'e}tape consid{\'e}r{\'e}e et reli{\'e}es entre elles
			par le m{\'e}canisme des <<~{\sl pipe}~>> gr{\^a}ce au caract{\`e}re <<~\verb=|=~>>,
	\item	soit nous utilisons une seule commande <<~{\tt sed}~>> devant ex{\'e}cuter
			ces deux requ{\^e}tes.
\end{itemize}

Dans un but p{\'e}dagogique, nous montrerons la seconde m{\'e}thode. En effet, elle mettra
en {\'e}vidence comment combiner les requ{\^e}tes entre-elles et aussi la fa\c{c}on de les
ordonner.

Examinons la requ{\^e}te {\`a} {\'e}crire pour ex{\'e}cuter la premi{\`e}re {\'e}tape. Tout d'abord, la commande
<<~{\tt sed}~>> associ{\'e}e {\`a} la suppression est <<~{\tt d}~>>. {\`A} cette commande,
la sp{\'e}cification des adresses de d{\'e}but et de fin doit correspondre {\`a} toute ligne commen\c{c}ant
par <<~\verb=#=~>>. Nous allons donc utiliser une expression r{\'e}guli{\`e}re et les deux
adresses seront identiques. L'expression r{\'e}guli{\`e}re ad{\'e}quate correspond {\`a}~:
\begin{enumerate}
	\item	d{\'e}but de ligne,
	\item	caract{\`e}re <<~\verb=#=~>>,
	\item	quelque chose {\'e}ventuellement, c'est-{\`a}-dire un caract{\`e}re quelconque un
			nombre quelconque de fois (z{\'e}ro {\'e}ventuellement).
\end{enumerate}
Les correspondances, au niveau des expressions r{\'e}guli{\`e}res sont~:\\
\begin{center}
\begin{tabular}{|@{\hspace{0.5cm}}l@{\hspace{0.5cm}}|@{\hspace{0.5cm}}l@{\hspace{0.5cm}}|}
	\hline
		\hfill Description \hfill	&
		\hfill Expression \hfill	\\
	\hline \hline
		d{\'e}but de ligne					&	\verb=^=		\\
		caract{\`e}re <<~\verb=#=~>>		&	\verb=#=		\\
		quelque chose {\'e}ventuellement	&	\verb=.*=		\\
	\hline
\end{tabular}
\end{center}

La requ{\^e}te correspondant {\`a} la premi{\`e}re {\'e}tape est donc~:
\begin{center}
\verb=/^#.*/d=
\end{center}

Examinons maintenant la requ{\^e}te {\`a} {\'e}crire pour ex{\'e}cuter la seconde {\'e}tape. Comme
il a {\'e}t{\'e} dit pr{\'e}c{\'e}demment, cette requ{\^e}te devra s'appliquer sur toutes les lignes
obtenues par la premi{\`e}re. Sachant que les requ{\^e}tes s'ex{\'e}cutent s{\'e}quentiellement,
lors de l'appel {\`a} la seconde, les enregistrements qui lui sont fournis sont
ceux qui r{\'e}sultent de la {\sl premi{\`e}re} ex{\'e}cution. Par cons{\'e}quent, les adresses
de d{\'e}but et de fin ne sont pas {\`a} sp{\'e}cifier car ce sont toutes les lignes
qu'il faut traiter.

La commande de substitution de <<~{\tt sed}~>> est <<~{\tt s}~>>. Sa syntaxe est~:
\begin{center}
{\tt s/}{\sl expression$_1$}{\tt /}{\sl expression$_2$}{\tt /}[{\tt g}]
\end{center}
Ici, <<~{\sl expression$_1$}~>> devra correspondre {\`a} la s{\'e}quence suivante~:
\begin{itemize}
	\item	caract{\`e}re <<~\verb=#=~>>,
	\item	quelque chose {\'e}ventuellement, c'est-{\`a}-dire une suite de caract{\`e}res
			quelconques un nombre quelconque de fois,
	\item	fin de ligne.
\end{itemize}
Quant {\`a} <<~{\sl expression$_2$}~>>, elle devra repr{\'e}sent{\'e} ce par quoi
<<~{\sl expression$_1$}~>> devra {\^e}tre remplac{\'e}, c'est-{\`a}-dire <<~{\bf rien}~>>.
Par cons{\'e}quent, {\bf <<~{\sl expression$_2$}~>> sera vide}. Il ne nous reste donc
plus qu'{\`a} formuler l'expression r{\'e}gul{\`e}re associ{\'e}e {\`a} <<~{\sl expression$_1$}~>>.
En fonction de ce que doit d{\'e}crire <<~{\sl expression$_1$}~>>, nous avons
les correspondances suivantes avec les expressions r{\'e}guli{\`e}res~:
\begin{center}
\begin{tabular}{|@{\hspace{0.5cm}}l@{\hspace{0.5cm}}|@{\hspace{0.5cm}}l@{\hspace{0.5cm}}|}
	\hline
		\hfill Description \hfill	&
		\hfill Expression \hfill	\\
	\hline \hline
		caract{\`e}re <<~\verb=#=~>>		&	\verb=#=	\\
		quelque chose {\'e}ventuellement	&	\verb=.*=	\\
		fin de ligne					&	\verb=$=	\\
	\hline
\end{tabular}
\end{center}

La requ{\^e}te correspondant {\`a} la seconde {\'e}tape est donc~:
\begin{center}
\verb=s/#.*$//=
\end{center}
Nous prenons donc bien en compte tout ce qu'il y a entre la premi{\`e}re occurence
du caract{\`e}re <<~\verb=#=~>> jusqu'{\`a} la fin de la ligne. Cette entit{\'e} est
substitu{\'e}e par <<~{\sl rien}~>>.

Nous obtenons donc~:
\begin{quote}
\begin{verbatim}
sed -e '/^#.*/d' -e 's/#.*$//' fichier
\end{verbatim}
\end{quote}

%%%%%%%%%%%%%%
\subsection{\label{adv-fltrs-sed-ex4}Exemple 4}

Le but de cet exemple est de r{\'e}cup{\'e}rer le champ <<~{\sl login}~>> et <<~{\sl UID}~>>
du fichier <<~{\tt /etc/passwd}~>>.

D'apr{\`e}s le format utilis{\'e} pour ce fichier (explicit{\'e} dans {\tt passwd(5)}), tous
les champs sont s{\'e}par{\'e}s par le caract{\`e}re <<~{\tt :}~>>. Par cons{\'e}quent,
le contenu de chaque champ interdit l'utilisation de ce caract{\`e}re. Les champs
de ce fichier auront donc comme propri{\'e}t{\'e}~:
\begin{quote}
Les caract{\`e}res contenus pour chaque champ du fichier {\tt /etc/passwd} sont
tous les caract{\`e}res alphanum{\'e}riques valides {\`a} l'exception du caract{\`e}re
<<~{\tt :}~>>.
\end{quote}
Pour repr{\'e}senter une cha{\^\i}ne de caract{\`e}res ne contenant pas <<~{\tt :}~>> {\`a}
l'aide des expressions r{\'e}guli{\`e}res, nous allons utiliser les plages de caract{\`e}res
en excluant les caract{\`e}res non valides. Nous aurons donc~:
\begin{center}
\verb=[^:]*=\\[1ex]
\end{center}

Les champs <<~{\sl login}~>> et <<~{\sl UID}~>> repr{\'e}sentent les premier et
troisi{\`e}me champ du fichier <<~{\tt /etc/passwd}~>>. Par cons{\'e}quent, la m{\'e}thode
{\`a} suivre pour d{\'e}crire notre requ{\^e}te est la suivante~:
\begin{itemize}
	\item	le d{\'e}but de ligne,\\[2ex]
	\item	une s{\'e}quence de caract{\`e}res ne contenant pas <<~{\tt :}~>> permettant
			d'isoler le premier champ, donc le champ <<~{\sl login}~>>,\\[2ex]
	\item	le caract{\`e}re <<~{\tt :}~>>,\\[2ex]
	\item	une s{\'e}quence de caract{\`e}res ne contenant pas <<~{\tt :}~>> permettant
			d'isoler le second champ,\\[2ex]
	\item	le caract{\`e}re <<~{\tt :}~>>,\\[2ex]
	\item	une s{\'e}quence de caract{\`e}res ne contenant pas <<~{\tt :}~>> permettant
			d'isoler le troisi{\`e}me  champ, donc le champ <<~{\sl UID}~>>,\\[2ex]
	\item	une s{\'e}quence de caract{\`e}res quelconques (<<~{\tt :}~>> pouvant y
			appara{\^\i}tre),\\[2ex]
	\item	la fin de la ligne.
\end{itemize}
Nous avons donc identifi{\'e} quatre groupements~:
\begin{itemize}
	\item	le premier associ{\'e} au champ <<~{\sl login}~>>,\\[2ex]
	\item	le second associ{\'e} {\`a} une s{\'e}quence de caract{\`e}res ne
			contenant pas <<~{\tt :}~>>, encadr{\'e}e {\`a} gauche et {\`a} droite par
			<<~{\tt :}~>>\footnote{Ce regroupement permet d'isoler le second
			champ avec les caract{\`e}res de d{\'e}limitation.},\\[2ex]
	\item	le troisi{\`e}me associ{\'e} au champ <<~{\sl UID}~>>,\\[2ex]
	\item	le quatri{\`e}me associ{\'e} {\`a} une s{\'e}quence de caract{\`e}res quelconques
			(<<~{\tt :}~>> pouvant y appara{\^\i}tre) jusqu'{\`a} la fin de la ligne.
\end{itemize}

Examinons maintenant l'expression r{\'e}guli{\`e}re associ{\'e}e~:
\begin{center}
\begin{tabular}{|@{\hspace{0.5cm}}l@{\hspace{0.5cm}}|@{\hspace{0.5cm}}l@{\hspace{0.5cm}}|}
	\hline
		\hfill Description \hfill	&
		\hfill Expression \hfill	\\
	\hline \hline
		d{\'e}but de ligne					&	\verb=^=		\\[2ex]
		s{\'e}quence de caract{\`e}res
		ne contenant pas <<~{\tt :}~>>	&	\verb=[^:]*=	\\[2ex]
		le caract{\`e}re <<~{\tt :}~>>		&	\verb=:=		\\[2ex]
		s{\'e}quence de caract{\`e}res
		ne contenant pas <<~{\tt :}~>>	&	\verb=[^:]*=	\\[2ex]
		le caract{\`e}re <<~{\tt :}~>>		&	\verb=:=		\\[2ex]
		s{\'e}quence de caract{\`e}res
		ne contenant pas <<~{\tt :}~>>	&	\verb=[^:]*=	\\[2ex]
		s{\'e}quence de caract{\`e}res
		quelconques						&	\verb=.*=		\\[2ex]
		fin de ligne					&	\verb=$=		\\
	\hline
\end{tabular}
\end{center}

En fonction des regroupements explicit{\'e}s pr{\'e}c{\'e}demment, nous obtenons
l'expression suivante~:
\begin{quote}
\begin{verbatim}
^\([^:]*\)\(:[^:]*:\)\([^:]*\)\(.*$\)
\end{verbatim}
\end{quote}

Maintenant, il ne nous reste plus qu'{\`a} prendre tout ce qui correspond
au premier et troisi{\`e}me regroupement dans chaque enregistrement (lignes
arrivant sur l'entr{\'e}e standard), et formater la sortie de la fa\c{c}on suivante~:
\begin{quote}
\begin{center}
\fbox{{\sl login}}~\tabkey~\fbox{{\sl UID}}
\end{center}
donc~:
\begin{center}
\fbox{{\sl regroupement$_1$}}~\tabkey~\fbox{{\sl regroupement$_3$}}
\end{center}
\end{quote}

Nous obtenons donc~:
\begin{quote}
\begin{verbatim*}
sed -e 's/^\([^:]*\)\(:[^:]*:\)\([^:]*\)\(.*$\)/\1\t\3/' /etc/passwd
\end{verbatim*}
\end{quote}

%%%%%%%%%%%%%%
\subsection{\label{adv-fltrs-sed-ex5}Exemple 5}

Le but de cet exemple est de copier de tous les utilisateurs pr{\'e}c{\'e}dent
l'entr{\'e}e du compte <<~{\tt uucp}~>> dans <<~{\tt /etc/passwd}~>> et
d{\'e}placer de les {\`a} la fin, compte <<~{\tt uucp}~>> compris. Le r{\'e}sultat sera
affich{\'e} sur la sortie standard.

Pour cela on fait appel {\`a} un fichier de commandes <<~{\tt sed}~>> que
l'on nommera <<~{\tt script.sed}~>>. L'appel se fera de la fa\c{c}on suivante~:
\begin{quote}
\begin{verbatim}
sed -f script.sed /etc/passwd
\end{verbatim}
\end{quote}


D{\'e}taillons les op{\'e}rations {\`a} effectuer~:
\begin{itemize}
	\item	{\sl Couper} dans un espace de travail particulier, {\`a} partir de
			la premi{\`e}re ligne jusqu'{\`a} celle commen\c{c}ant par <<~{\tt uucp}~>>.
	\item	Se placer {\`a} la fin de l'espace m{\'e}moire contenant les enregistrements
			{\`a} traiter par <<~{\tt sed}~>> et {\sl coller} le contenu de l'espace
			pr{\'e}c{\'e}dent.
\end{itemize}

Pour cela, nous allons utiliser les commandes li{\'e}es {\`a} la gestion de l'espace
m{\'e}moire et l'espace de travail de <<~{\tt sed}~>>. Comme il l'a {\'e}t{\'e} d{\'e}crit
{\`a} la section \ref{sed-fonct}, <<~{\tt sed}~>> dispose d'un espace m{\'e}moire
permettant de stocker temporairement, un certain nombre d'informations.

La premi{\`e}re {\'e}tape consiste donc {\`a} partir du d{\'e}but du fichier (donc de la
premi{\`e}re ligne) et parcourir jusqu'{\`a} la premi{\`e}re ligne commen\c{c}ant par
<<~{\tt uucp}~>>. Sachant que les nom d'utilisateurs sont uniques, le
fichier ne contiendra qu'une et une seule ligne commen\c{c}ant par
<<~{\tt uucp}~>>. Les adresses de d{\'e}but et de fin sont donc~:
\begin{center}
\begin{tabular}{|@{\hspace{0.5cm}}l@{\hspace{0.5cm}}|@{\hspace{0.5cm}}l@{\hspace{0.5cm}}|}
	\hline
		\hfill Description \hfill	&
		\hfill Expression \hfill	\\
	\hline \hline
		Premi{\`e}re ligne									&	\verb=1=		\\
		Premi{\`e}re ligne commen\c{c}ant par <<~{\tt uucp}~>>	&	\verb=/^uucp/=	\\
	\hline
\end{tabular}
\end{center}
La commande <<~{\tt sed}~>> appropri{\'e}e pour stocker l'ensemble de ces lignes
dans un espace m{\'e}moire est <<~{\tt H}~>> (cf. section \ref{sed-cmds}).
La requ{\^e}te associ{\'e}e est donc~:
\begin{quote}
\begin{verbatim}
1,/^uucp/H
\end{verbatim}
\end{quote}

{\`A} ce stade de l'ex{\'e}cution, l'enregistrement courrant arrivant sur l'entr{\'e}e de
<<~{\tt sed}~>> est la ligne du fichier <<~{\tt /etc/passwd}~>> d{\'e}finissant le
compte utilisateur suivant celui de <<~{\tt uucp}~>>. Aucune action n'est sp{\'e}cifi{\'e}e.
Les lignes sont donc r{\'e}{\'e}critures sans aucune op{\'e}ration sur la sortie standard.

Lorsque la fin de fichier est rencontr{\'e}e, <<~{\tt sed}~>> doit afficher ce
qu'il a pr{\'e}c{\'e}demment m{\'e}moriser dans son espace m{\'e}moire. Par cons{\'e}quent, l'adresse
valide pour ex{\'e}cuter la requ{\^e}te est celle correspondant {\`a} la fin de fichier~:
<<~\verb=$=~>>. Ici les adresses de d{\'e}but et de fin sont identiques~: elles
doivent correspondre toutes les deux {\`a} la fin de fichier.
La commande <<~{\tt sed}~>> appropri{\'e}e pour afficher sur la sortie standard
l'ensemble de ces lignes pr{\'e}c{\'e}demment stock{\'e}es dans un espace m{\'e}moire est
<<~{\tt G}~>> (cf. section \ref{sed-cmds}).
La requ{\^e}te associ{\'e}e est donc~:
\begin{quote}
\begin{verbatim}
$G
\end{verbatim}
\end{quote}

En conclusion, le contenu du fichier <<~{\tt script.sed}~>> est~:
\begin{quote}
\begin{verbatim}
1,/^uucp/H
$G
\end{verbatim}
\end{quote}

\begin{remarque}
L'espace m{\'e}moire utilis{\'e} est, bien s{\^u}r, propre {\`a} <<~{\tt sed}~>>. Les
op{\'e}rations <<~{\tt g}~>> et <<~{\tt h}~>> sont similaires mais avec un espace
vide au d{\'e}part.
\end{remarque}


%%%%%%%%%%%%%%%%%%%%%%%%%%%%%%%%%%%%%%%%%%%%%%%%%%%%%%%%%%%%
\section{\label{adv-fltrs-sed-f}Remarque sur l'utilisation de
			l'option <<~{\tt -f}~>>}

{\`A} la r{\`e}gle 10 explicit{\'e}e {\`a} la section \ref{pgm-intro-basic-rules}, nous
avions pr{\'e}ciser que la premi{\`e}re ligne d'un script devait sp{\'e}cifier le nom
de l'ex{\'e}cutable associ{\'e} au shell utiliser.

Nous avons vu de m{\^e}me, que la commande <<~{\tt sed}~>> dispose de l'option
<<~{\tt -f}~>>, permettant de pr{\'e}ciser un fichier de requ{\^e}tes <<~{\tt sed}~>>.

Si l'analogie est faite avec le shell, le processus charg{\'e} d'{\'e}valuer chaque
ligne du fichier est l'ex{\'e}cutable sp{\'e}cifi{\'e} au niveau de la premi{\`e}re
ligne du fichier contenant les instructions. Par cons{\'e}quent si le fichier
contenant les requ{\^e}tes <<~{\tt sed}~>>~:
\begin{itemize}
	\item	a comme premi{\`e}re ligne <<~\verb=#!/usr/bin/sed=~>>\footnote{Nous
			consid{\'e}rons ici que l'ex{\'e}cutable de <<~{\tt sed}~>> se trouve dans
			le r{\'e}pertoire <<~{\tt /usr/bin}~>>. Pour en conna{\^\i}tre la
			localisation exacte sur votre syst{\`e}me, vous pouvez utiliser
			la commande <<~{\tt which(1)}~>> ou <<~{\tt whereis(1)}~>>.},
	\item	est accessible en ex{\'e}cution,
\end{itemize}
il peut {\^e}tre consid{\'e}r{\'e} comme un filtre ex{\'e}cutant les requ{\^e}tes <<~{\tt sed}~>>
qu'il contient.

Par exemple, si cette notion est appliqu{\'e}e au fichier d{\'e}crit dans l'exemple
de la section \ref{adv-fltrs-sed-ex5}, le contenu de ce fichier
sera~:
\begin{quote}
\begin{verbatim}
#!/usr/bin/sed
1,/^uucp/H
$G
\end{verbatim}
\end{quote}

La commande qui sera saisie deviendra~:
\begin{quote}
\begin{verbatim}
script.sed /etc/passwd
\end{verbatim}
\end{quote}
